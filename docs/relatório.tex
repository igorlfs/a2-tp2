\documentclass{article}

\usepackage{hyperref}               % Use hyperlinks

\usepackage[T1]{fontenc}            % Codificação para português 
\usepackage[portuguese]{babel}      % Português


\usepackage{algorithm}              % Pseudocódigo
\usepackage[noend]{algpseudocode}

\usepackage{graphicx}               % Coloque figuras
\usepackage{float}                  % Figuras no lugar "esperado"
\graphicspath{{./images/}}          % Localização das imagens

\usepackage{amsmath}                % Use equações alinhadas

\usepackage{enumitem}               % Corrige indentação dentro de enumerates/itemizes
\setlist{  listparindent=\parindent, parsep=0pt, }

\usepackage{hyphenat}               % Use hifens corretamente
\hyphenation{mate-mática recu-perar}

\usepackage[backend=biber, style=authoryear-icomp]{biblatex}
\addbibresource{references.bib}
\usepackage{csquotes}               % https://tex.stackexchange.com/a/229653

\author{Igor Lacerda Faria da Silva}
\title{Trabalho Prático II \\
Algoritmos II}
\date{}

\begin{document}

\maketitle

\section{Introdução}

O Trabalho Prático 2 da disciplina de Algoritmos II possui como proposta a análise de diferentes algoritmos para resolver o Problema do Caixeiro Viajante (PCV), com algumas restrições. Em suma, foi implementado um gerador de instâncias do problema, que são submetidas aos três algoritmos desenvolvidos (\textit{Twice Around The Tree}, Christofides e \textit{Branch And Bound}) e suas métricas de execução coletadas e analisadas.

As instâncias do PCV seguem a restrição de possuir como função de custo uma métrica:
\begin{itemize}

	\item \( c(u,v) \geq 0 \)

	\item \( c(u,v) = 0  \Leftrightarrow  u = v \)

	\item \( c(u,v) = c(v,u) \) (Simetria)

	\item \( c(u,v) \leq c(u,w) + c(w,v) \) (Desigualdade Triangular)

\end{itemize}

As métricas usadas são a distância euclidiana, definida para \( P = (p_x,p_y) \land Q=(q_x,q_y) \) como:

\[ d_{\textrm{euclidiana}} = \sqrt{ (p_x-q_x)^{2} + (p_y-q_y)^2 } \]

E a distância de Manhattan, definida como:

\[ d_{\textrm{Manhattan}} = | p_x - q_x | + | p_y - q_y | \]


\section{Implementação}

O trabalho foi implementado na linguagem Python, versão 3.10.8, no sistema operacional Linux. O programa segue o paradigma de programação procedural, visto que não há uma distinção muito clara de quais seriam as classes em uma abordagem de programação orientada a objetos. Dito isso, foi implementada uma única classe. O programa foi testado usando a biblioteca \texttt{pytest}.

\subsection{Arquivos}
O programa está dividido em 3 módulos: \texttt{calculate}, \texttt{generators}, \texttt{algorithms}, sendo os dois primeiros de auxílio ao terceiro, que implementa de fato os algoritmos. Além disso, existe um módulo de testes que engloba os testes dos outros módulos.

\subsubsection{Calculate}

O módulo \texttt{calculate} é o mais simples, tendo como propósito apenas o cálculo de algumas medições úteis. Possui duas funções: uma para calcular a distância de um conjunto de pontos usando uma determinada métrica (que pode ser ou a distância euclidiana ou a distância de Manhattan). A outra função recebe um ciclo e um grafo e retorna o custo de se percorrer esse caminho.

A função implementada pela biblioteca \texttt{networkx} para o algoritmo de Christofides retorna uma lista de vértices, que inclui o vértice de partida duas vezes (uma vez para fechar o ciclo). Para manter a compatibilidade com essa decisão da biblioteca, a função \texttt{calculate\_cost()} assume que o caminho possui o vértice inicial igual ao vértice final.

\subsubsection{Generators}

O módulo \texttt{generators} possui duas funções principais e uma função auxiliar. Seu propósito é, naturalmente, gerar instâncias do PCV. A função \texttt{generate\_points()} gera um conjunto de \( n \) pontos no plano cartesiano, entre um piso e teto passados como parâmetros. As coordenadas dos pontos são números inteiros, pela especificação. A função \texttt{generate\_instances()} cria um grafo do PCV usando as funções \texttt{generate\_points()} e \texttt{calculate\_distance()}.

% TODO: measure_algorithm() ?

\subsubsection{Algorithms}

O módulo principal do programa é baseado nos três algoritmos para solução do PCV: algoritmo de Christofides, \textit{Branch And Bound}, \textit{Twice Around The Tree}. Também existe uma camada de abstração que facilita a execução em escala.

\begin{itemize}

	\item \textit{Twice Around The Tree}

	      O algoritmo de implementação mais fácil foi o \textit{Twice Around The Tree}, em particular pois foi permitido o uso da biblioteca \texttt{networkx} para fazer o cálculo da árvore geradora e o caminhamento pré-ordem dos vértices. Esse algoritmo é aproximativo e explora as árvores geradoras mínimas (AGM) como ponto de partida: é computacionalmente simples calcular a AGM e é esperado que o circuito hamiltoniano mínimo compartilhe \textit{pelo menos} algumas arestas. Com a AGM em mãos, o grafo é percorrido usando uma DFS para construir o caminho de menor custo a partir desta, excluindo repetições.

	\item Christofides

	\item \textit{Branch And Bound}

\end{itemize}

\section{Análise Experimental}

\section{Conclusão}

% \printbibliography

\end{document}
